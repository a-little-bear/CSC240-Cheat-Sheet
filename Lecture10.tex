\documentclass[12pt]{alittlebear}

\def\name{Joseph Siu}
\def\course{CSC240}
\def\headername{Lecture }
\def\headernum{10}
%CSC240 Commands
%%% Logic
\newcommand{\nnot}{\mathrm{NOT}}
\newcommand{\aand}{\,\,\mathrm{AND}\,\,}
\newcommand{\oor}{\,\,\mathrm{OR}\,\,}
\newcommand{\iimplies}{\,\,\mathrm{IMPLIES}\,\,}
\newcommand{\xor}{\,\,\mathrm{XOR}\,\,}
\newcommand{\iif}{\,\,\mathrm{IFF}\,\,}
\newcommand{\nand}{\,\,\mathrm{NAND}\,\,}
%%% Sets
\newcommand{\nats}{\mathbb{N}}
\newcommand{\ints}{\mathbb{Z}}


\usepackage{changepage}

\begin{document} 

\newm{
    A set $C$ is countable if it is empty or there is a surjective function from $\N$ to $C$.

    If $A,B$ are countable, so is $A\cup B$. 

    If $A$ is countable and $B\subseteq A$, then $B$ is coutable.

    If $A,B$ are countable, so is $A\times B$.

    If $A\neq\nil$ and countable and there is a surjective function from $A$ to $B$ then $B$ is countable. 

    \newp{
        There is a surjective function $g:\N\to A$ . Then $h:\N\to B$ is a surjective function where $h(i)=f(g(i))$ for all $i\in\N$. 
    }
}

\newt{thm1}{
    $\Q^+$ is countable;

    \newp{
        Let $f:\Z^+\times \Z^+\to\Q^+$ be defined by $f(a,b)=\frac{a}{b}$.

        Note that $(1,2),(2,4),(3,6)$ all map to $\frac12$ under $f$. 

        Since $f$ is surjective, thus $\Q^+$ is countable.
    }
}

\newt{thm2}{
    $\{0,1\}^*$ = the set of all finite inary sequences is countable.

    a lexicographically order: $\lambda, 0, 1, 00, 01, 10, 11, 000, \ldots$.

    $g: \N\times\N\to\{0,1\}^*$, $g(i,j)$ = ``$j^{th}$ lexicographically smallest string of length $i$ if $0\leq j\leq 2^i$; otherwise $\lambda$''

    \newm{
        There are $2^i$ binary string of length $c$.

        \begin{tabular}{c|c|c|c|c|c|c}
            g & 0 & 1 & 2 & 3 & 4 & \\
            \hline 0 & $\lambda$ & $\lambda$ & $\lambda$ & $\lambda$ & $\lambda$ & $\cdots$ \\
            \hline 1 & 0 & 1 & $\lambda$ & $\lambda$ & $\lambda$ & $\cdots$ \\
            \hline 2 & 00 & 01 & 10 & 11 & $\lambda$ & $\cdots$ \\
            \hline 3 & 000 & & & & &
        \end{tabular}
    }
}

\newt{thm3}{
    The set of all finite strings of ASCII characters is countable.
}

\newco[Corollary of Theorem 3]{co1}{
    The set of all syntactically correct Python programs is countable. 
}

\np
\newt{thm4}{
    A countable union of countable sets is countable
    \newp{
        Suppose $C$ is a countable collection of countable sets.

        Then there is a surjective function $f:\N\to C$ [$f(i)$ is a set in $C$] and 

        for each $S\in G$, there is a surjective function on $g_s:\N\to S$

        (now we want to prove $\bigcup\{S\mid S\in C\}=\bigcup\{f(i)\mid i\in\N\}$)

        Consider the function $h:\N\times\N\to\bigcup\{S\mid S\in C\}$ such that $h(i,j)=g_{f(i)}(j)$ for the $j^{th}$ element of $f(i)$.

        Let $X\in \bigcup\{S\mid S\in C\}$ be arbitrary;
        
        Then there exists $S\in C$ such that $x\in S$, and there exists $i\in\N$ such that $f(i)=S$.  

        Since $x\in S$ and $g_s$ is surjective, there exist $j\in\N$ such that $g_s(j)=x$.He

        nce $x=g_{f(i)}(j)=h(i,j)$, so $h$ is surjective.

        Thus $\bigcup\{S\mid S\in C\}=\bigcup\{f(i)\mid i\in\N\}$ is countable.  
    }
}

\newt{thm5}{
    The set $\{0,1\}^\omega$ of all infinite binary sequences is uncountable MAT 8.1.4.

    \newp{
        Suppose $\{0,1\}^\omega$ is countable. Then there exists a surjective function $f:\N\to\{0,1\}^\omega$.

        $B(0)= 1 0 0 0 1 1 \cdots$

        $B(1) = 0 1 1 1 0 1 \cdots$

        $B(2) = 1 0 0 0 0 0 \cdots$

        $\vdots$

        $B(i)=\{B(i)_j\}_{j\geq0}$

        Let $D$ = the sequence of bits on the diagonal. That is, $D_i = B(i)_i$ for all $i\in\N$.

        Let $C$ = the sequence of bits obtained form $D$ by complementing evrry bit. So, $C_i=1-B(i)_i$

        $D = 1 1 0 \cdots$

        $C = 0 0 1 \dots$

        $C\in\{0,1\}^\omega$ and $B$ is surjective so $\exists j\in\N$ such that $B(j)=C$. Then $B(j)_j=C_j=1-B(j)_j$. This is a contradiction.
    }
}

\newm{
    Another example of a diagonalization proof. 

    There is a compile (program) $C$ that determines whether a given ASCII string $P$ is a syntactically correct Python program that makes a single ASCII string as input i.e. $P\in C$.

    $G(P)$ outputs True if $P\in C$, False if $P\notin C$.

    Want a python program $H$ that takes as input two ASCII strings, $P$ and $X$ such that $H(P,X)$ outputs True if $P\in C$ and $P(x)$ halts False if $P\notin C$ or $P(X)$ runs forever. 
}

\np
\newt[Halting Problem]{thm5}{
    No such python program $H$ exists. 

    \newp{
        Suppose there has such a python program $H$.

        def t1(P: str, x: str):

        $\vdots$

        If justreturn is the string

        justreturn = ``def j(s: str): return $s'' \in C$, then $H$(justreturn, 'hello') returns True

        if goforever is the string

        goforever = ``def g(t: str): while True: pass'' $\in C$, then $H$(goforever, 'hello') returns False

        \dots

        Consider the syntatically correct Python function $D$:


        \newn{
            def D(x):
                    if H(x,x):
                        while True: pass
                    else:
                        return True
        }

        
        let $d\in C$ be the string, \newn{ d = def function D(x): if H(x,x): while True: pass else: return True}

        What happens when Drunson inputed from the code of D:

        \indenv{
            If $H(d,d)$ = false then $D(d)$ returns True

            If $H(d,d)$ = true then $D(d)$ goes into a infinite loop


        }

        From the definition of $H$,

        \indenv{
            If $D(d)$ returns then $H(d,d)$=True

            If $D(d)$ goe sinto an infinite loop then $H(d,d)$ = False
        }

        This is a contradiction.

    }
}





\end{document}