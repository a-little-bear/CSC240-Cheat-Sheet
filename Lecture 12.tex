\documentclass[11pt, cyan, night, 0.5in]{hw}

\def\course{CSC240}
\def\headername{Lecture 11 Week 8}
\def\name{Joseph Siu}
\def\email{joseph.siu@mail.utoronto.ca}
\def\logo{clsfiles/qunwang}

\usepackage{clsfiles/csc240}

\begin{document}

\coverpage[clsfiles/stars]

For $n\in\Z^+$, let $M(n)=\begin{cases}
    c & \text{if } n=1\\
    M(\ceil{\frac{n}{2}}) + M(\floor{\frac{n}{2}}) + dn & \text{if } n>1
\end{cases}.$

For all $i\in\N$, let $Q(i)=``M(2^i)=c2^i+di2^i"$.

\newcl{1}{
    $\forall i\in\N. Q(i)$.
}

\newt{1}{
    $M(n)\in\Th(n\log n)$.
}

\newl{1}{
    $\forall n\in\Z^+. \forall m\in\Z^+. (m\leq n) \iimplies (M(m)\le qM(n)).$

    That is, $M$ is a non-decreasing function.
}

\newp{[Proof of Lemma \ref{lemma:l1} by induction on $n$]\hfill

    \indenv{
        For $n\in\Z^+$. Let $R(n)=``\forall m\in\Z. (m\leq n)\iimplies (M(m)\leq M(n))$''

        Let $n\in\Z^+$ be arbitrary, suppose $R(n')$ is true for all $n'\in\Z^+$ such that $n'<n$;
        \indenv{
            $M(1)\le M(1)$, $M(2)\le M(2)$, and $M(1)=c\le 2c+2d=M(2)$ since $c,d\ge 0$.
        }
        So $R(1)$ and $R(2)$ are true.
        \indenv{
            Consider $n\ge 2$

            Then $1\le \floor{\frac{n}{2}}\le \ceil{\frac{n}{2}}\le n-1 < n$
        }

        So $R(\floor{\frac{n}{2}})$, $R(\ceil{\frac{n}{2}})$, and $R(n-1)$ all hold
        \indenv{
            Let $m\in\Z^+$ be arbitrary and suppose $m\le n$.

            \begin{proofcases}
                \case $m=n$.
                \indenv{
                    $M(m)=M(n)$ by substitution
                }
                \case $m=n-1$. 
                \indenv{
                    \begin{align*}
                        \Aleq{M(n-1)}{M(\ceil{\frac{n-1}{2}})+M(\floor{\frac{n-1}{2}}+d(n-1))}{since $n-1>1$}
                        \Alleq{}{M(\ceil{\frac{n}{2}})}{since $\ceil{\frac{n-1}{2}}\le\ceil{\frac{n}{2}}\le n$}
                        &+M(\floor{\frac{n}{2}}) & \T{since $\floor{\frac{n-1}{2}}\le\floor{\frac{n}{2}}\le n$} &&\\
                        &+dn & \T{since $d\ge 0$} &&\\
                        &= M(n)
                    \end{align*}
                }
                \case $m< n-1$.
                \indenv{
                    Then $M(m)\le M(n-1) \le M(n)$ by induction hypothesis
                }

                In all cases $M(m)\leq M(n)$

                So $(m\le n)\iimplies (M(m)\le M(n))$.
            \end{proofcases}
        }
        Since $m$ was arbitrary, $R(n)$ holds.
    }
    By strong induction, $\forall n\in\Z^+. R(n)$.
}

\newp{[Proof of Theorem \ref{theorem:t1}]\hfill
    
    \indenv{
        Let $n\in\Z^+$ be arbitrary, assume $n\geq 2$.

        Let $2^i$ be the smallest power of $2$ that is greater or equalt to $n$

        Then, $n\le 2^i<2n$ which implies $i<\log_2(2n)$ and so \begin{align*}
            M(n)\leq M(2^i) &=c2^i+di2^i\\
            &<2cn+2dn\log_2(2n)\\
            &=2cn+2dn(1+log_2(n))\\
            &<(2c+4d)n\log_2 n
        \end{align*}
    }

    By generalization, we have $\forall n\in\N. \sqrbra{(n\geq2)\iimplies M(n)\leq (2c+4d)n\log_2 n}$, so $M(n)\in O(n\log n)$.

    \indenv{
        Let $2^j$ be the largest power of 2 that is less than or equal to $n$.

        Then, we have $n\geq 2^j> \frac{n}{2}$ and so \begin{align*}
            M(n)&\geq M(2^j) \\
            &=c2^j+dj2^j\\
            &>c\frac{n}{2}+\frac{dn}{2}\log_2 (\frac{n}{2})\\
            &\ge \frac{dn}{4} \log_2 n \quad \T{for $n\ge 4$}
        \end{align*}
    }

    Thus, $M(n)\in\Om(n\log n)$

    Therefore, we conclude $M(n)\in\Th(n\log n)$.
}

\section*{Analyzing the Worst Case Time Complexity of Iterative Algorithms}

\begin{itemize}
    \item Code without loops or procedure calls - $O(1)$ step
    \item Loops: if a loop is executed $O(f(n))$ times and each iteration takes $O(g(n))$ steps, then the entire loop takes $O(f(n)\cd g(n))$ steps.
    \item If statements: Suppose $A,B,C$ are blocks of code that take $O(f(n)), O(g(n)), O(h(n))$ steps respectively. Then 

    \begin{algorithm}
        \begin{algorithmic}
            \aif{$A$}
                \as $B$
            \aelse
                \as $C$
            \aeif
        \end{algorithmic}
    \end{algorithm}

    takes $O(f(n) + \max\{g(n),h(n)\})$ steps.
    \item Procedure + function calls: Suppose that the worst case step complexity of a procedure (or function) $\cal{P}$ with input of size $r$ is $T(r)\in O(f(r))$, then a call to $\cal{P}$ with an input of size $g(n)$ takes $O(f(g(n)))$ steps.
\end{itemize}


\begin{algorithm}
Consider the following Merge code:

    \begin{algorithmic}
        \afunc{Merge}{$A[1..m],m,B[1..n],n$}

        \as $i\gets 1$
        \as $j\gets 1$
        \as $h\gets 1$\\
    
        \awh{$i\leq m$ and $j\leq n$}
            \aif{$A[i]<B[j]$}
                \as $C[h]\gets A[i]$
                \as $i\gets i+1$
            \aelse
                \as $C[h]\gets B[j]$
                \as $j\gets j+1$
            \aeif
            \as $h\gets h+1$
        \aewh\\
    
        \aif{$i>m$}
            \awh{$j\leq n$}
                \as $C[h]\gets B[j]$
                \as $j\gets j+1$
                \as $h\gets h+1$
            \aewh
        \aelse 
            \awh{$i\leq m$}
                \as $C[h]\gets A[i]$
                \as $i\gets i+1$
                \as $h\gets h+1$
            \aewh
        \aeif
        \aefunc
    \end{algorithmic}

    \tbf{Input size}: $m,n$ (better than $\max\{m,n\}, m+n$)

    \tbf{Step}: ``increasement of $i+j$'' counting steps the exact same way as ``assignment to $C$ ($m+n)$''. We can also do ``comparison between elements of $A$ and elements of $B$'' (not as good as the previous 2).

    \tbf{Time Complexity}: $m+n-1$.
\end{algorithm}

\np

Step = ``\# of iterations of 1st while loop.''

1st while loop terminates as soon as $i>m$ or $j>n$, so either $i$ has been increased $m$ times or $j$ has been increased $n$ times.

Exactly one of $i$ and $j$ is increased each iteration, so both conditions can't be true.

In the first case $j$ is increased at most $n-1$ times.

In the second case $i$ is increased at most $m-1$ times.

In both cases $\le m+n-1$ comparisions are performed.

If $T_{M\in\T{RUR}}(m,n)$ to the worst case number of comparisions \underline{between elements of $A[1..m]$ and $B[1..n]$} then $T_{M\in\T{RUR}}(m,n)\le m+n-1$.

\tit{Why is $m+n-1$ a lower bound?}

For each $m.n$ what are the examples of $A[1..m]$ and $B[1..n]$ for which $n+m-1$ comparisions are performed?

If all elements in $A$ and $B$ are less than $A[m]$ and except for $A[m]$ are less than $B[n]$.

\[A=\underbrace{[0\;\cdots\;0\;2]}_{n}\]
\[B=\underbrace{[0\;\cdots\;0\;1]}_{n}\]
\[T_{\T{Merge}}(m,n)\geq m+n-1.\]

\begin{algorithm}
    \begin{algorithmic}
        \afunc{MergeSort}{$A[1..n],n$} \Comment{$n$ = input size}
        \aif{$n\le 1$} \Comment{step = comparision between elements of $A$}
            \as return
        \aeif
        \as $m\gets \ceil{\frac{n}{2}}$
        \as $A'\gets A[1..m]$
        \as $A''\gets A[m+1..n]$
        \as $M\in \T{RURSORT}(A', m)$
        \as $M\in \T{RURSORT}(A'', n-m)$
        \as $A\gets \T{Merge}(A[1..m],m,A[m+1..n],n-m)$
        \aefunc
    \end{algorithmic}
\end{algorithm}

\indenv{
    Let $M:\Z^+\to\N$ denote the worst case time complexity of MergeSort

    \indenv{
        $M(n)=0\quad$ if $n\le 1$

        $M(n)\leq M(\ceil{\frac{n}{2}})+M(\floor{\frac{n}{2}})+n-1\quad$ if $n>1$ 

        \footnote{\tbf{Note:} We also need to show the linear bound is achievable}Let $M':\Z^+\to\N$ be $M'(n)=\begin{cases}
            0 & \T{if } n\le 1\\
            M'(\ceil{\frac{n}{2}})+M'(\floor{\frac{n}{2}})+n-1 & \T{if } n>1
        \end{cases}$

        Then $M'(n)\in O(n\log n)$ as proven at the start.       
        
        It is very easy to prove by induction $\forall n\in\Z^+. M(n)\le M'(n)$.
    }

    Hence $M(n)\in O(n\log n)$.

}

\end{document}