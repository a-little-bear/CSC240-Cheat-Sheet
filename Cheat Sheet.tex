\documentclass{homework}
\author{Joseph Siu}
\class{CSC240}
\date{\today}
\title{Cheat Sheet}

%CSC240 Commands
%%% Logic
\newcommand{\nnot}{\mathrm{NOT}}
\newcommand{\aand}{\,\,\mathrm{AND}\,\,}
\newcommand{\oor}{\,\,\mathrm{OR}\,\,}
\newcommand{\iimplies}{\,\,\mathrm{IMPLIES}\,\,}
\newcommand{\xor}{\,\,\mathrm{XOR}\,\,}
\newcommand{\iif}{\,\,\mathrm{IFF}\,\,}
\newcommand{\nand}{\,\,\mathrm{NAND}\,\,}
%%% Sets
\newcommand{\nats}{\mathbb{N}}
\newcommand{\ints}{\mathbb{Z}}

%Other Commands
\newcommand{\Set}[1]{\{#1\}}
\newcommand{\T}[1]{\text{#1}}
\newcommand{\Al}[3]{#1 &=#2 &\text{#3}&&\\}
\newcommand*{\eg}{\leavevmode\unskip , e. g., \ignorespaces} % for example
\newcommand*{\ie}{\leavevmode\unskip, i. e., \ignorespaces} % that is
\newcommand{\nil}{\varnothing}
\AtBeginDocument{\def\O{\cal{O}}} % Big Oh
\AtBeginDocument{\def\C{\bb{C}}} % Complex
\newcommand{\R}{\bb{R}} % Reals
\newcommand{\Q}{\bb{Q}} % Rationals
\newcommand{\Z}{\bb{Z}} % Integers
\newcommand{\N}{\bb{N}} % Naturals
\renewcommand{\P}{\bb{P}} % Primes
\newcommand{\Pset}[1]{\mathcal{P}(#1)} %power set
\newcommand{\Relate}[2]{#1\mathcal{R}#2} %relation
\newcommand{\relate}{\mathcal{R}}
\newcommand{\F}{\bb{F}} 
\newcommand{\GF}[1][2]{\bb{F}_{#1}} 
\newcommand{\modulo}[1][n]{\Z/#1\Z} 
\newcommand{\ra}{\rightarrow}
\newcommand{\Ra}{\Rightarrow}
\newcommand{\?}{\stackrel{?}{=}}
\newcommand{\is}{\equiv}
\newcommand{\al}{\alpha}
\newcommand{\ep}{\varepsilon}
\renewcommand{\phi}{\varphi}
\newcommand{\p}{\partial}
\newcommand{\injective}{\hookrightarrow}
\newcommand{\surjective}{\twoheadrightarrow}
\newcommand{\bijective}{\hookrightarrow\mathrel{\mspace{-15mu}}\rightarrow}
\newcommand{\derivative}[2][x]{\frac{\D #2}{\D #1}}
\newcommand{\ceil}[1]{\left\lceil#1\right\rceil}
\newcommand{\floor}[1]{\left\lfloor#1\right\rfloor}
\newcommand{\near}[1]{\left\lfloor#1\right\rceil}
\newcommand{\arr}[1]{\left\langle#1\right\rangle}
\newcommand{\paren}[1]{\left(#1\right)} %pair / ()
\newcommand{\brk}[1]{\left[#1\right]} %[]
\newcommand{\abs}[1]{\left|#1\right|}
\newcommand{\curl}[1]{\left\{#1\right\}} %set {}
\newcommand{\func}[3]{#1: #2 \rightarrow #3}

%Environments
\theoremstyle{definition}
\newtheorem*{claim}{Claim}
\newtheorem{definition}{Definition}
\newtheorem{theorem}{Theorem}
\newtheorem{lemma}{Lemma}

%Sorted List
\usepackage{datatool}% http://ctan.org/pkg/datatool
\newcommand{\sortitem}[1]{%
  \DTLnewrow{list}% Create a new entry
  \DTLnewdbentry{list}{description}{#1}% Add entry as description
}
\newenvironment{sortedlist}{%
  \DTLifdbexists{list}{\DTLcleardb{list}}{\DTLnewdb{list}}% Create new/discard old list
}{%
  \DTLsort{description}{list}% Sort list
  \begin{itemize}%
    \DTLforeach*{list}{\theDesc=description}{%
      \item \theDesc}% Print each item
  \end{itemize}%
}

\begin{document} %\maketitle

\section{Predicate and propositional logic}

\subsection{Terminologies}

\begin{sortedlist}
    \sortitem{\textbf{Proposition} = statement either \textit{true} or \textit{false}}
    \sortitem{\textbf{Predicate} = \textit{proposition} whose truth depends on \textit{variables} / function with range \textit{\{T,F\}}}
    \sortitem{\textbf{Propositional Formula} = expression built from \textit{Boolean variables} using \textit{connectives} with \textbf{no} \textit{predicates} or \textit{quantifiers}}
    \sortitem{\textbf{Truth Assignment} = function from a set of \textit{propositional variables} to \textit{\{T,F\}}; a row of the truth table}
    \sortitem{\textbf{Satisfiability Problem (SAT)} = output \textit{YES} if the \textit{propositional formula} is \textit{satisfiable}, vice versa}
    \sortitem{\textbf{Disjunctive Normal Form (DNF)} = a \textit{disjunction} of \textit{conjunctions} of \textit{literals}}
    \sortitem{\textbf{Conjunctive Normal Form (CNF)} = a \textit{conjunction} of \textit{disjunctions} of \textit{literals}}
    \sortitem{\textbf{Predicate Logic Formula} = \textit{predicates} (fixed number of arguments) + \textit{connectives} + \textit{quantifiers}}
    \sortitem{\textbf{Interpretation} = a case with non-empty domain; free variables with assigned domain elements; function from relevant domain to range}
    \sortitem{\textbf{Prenex Normal Form} = [some quantifications]+[formula without quantifiers]}
\end{sortedlist}

Trivial Terms:

\begin{itemize}
    \item Connectives = negation, conjunction, disjunction, exclusive-or, implication, equivalence
    \item Universal / Existential quantification 
    \item Boolean variable = variables that are either True of False (does not depend on other variables like predicate does)
    \item Truth table = $n$ variables truth table has $2^n$ rows\\
    \item Tautology / Valid = \textbf{propositional formula} which all entries are \textit{True} / \textbf{predicate logic formula} which all \textit{interpretations} are true
    \item Unsatisfiable / Contradiction = \textbf{propositional formula} which all entries are \textit{False} / \textbf{predicate logic formula} is false for all \textit{interpretation}
    \item Satisfiable = \textit{propositional formula} which at least one entry is \textit{True} / \textbf{predicate logic formula} is true for some \textit{interpretation}\\
    \item \textbf{P} = all decision problems can be solved in polynomial time
    \item \textbf{NP} = all decision problems can be verified in polynomial time; SAT $\in$ NP
    \item Literal = variable or the negation of it
    \item Clause = disjunction of literals
    \item CNF-SAT = \textit{SAT} but \textit{propositional formula} in \textit{CNF}\\
    \item Constant = a particular element in a domain
    \item Variable = any element in a domain
    \item quantified = variable with quantifier
    \item unquantified / free = variables that are not quantified
    \item valuation = maps free variable to domain element
    \item logically implies (equivalent) = P's interpretation is True => Q is also True
\end{itemize}

\subsection{Subtle details}

\begin{enumerate}
    \item Quantifiers are flipped with Negation / Hypothesis of implication
    \item Order of quantifiers does matter
    \item Every propositional formula is equivalent to a \textit{DNF / CNF}
    \item To construct a DNF form, for each line with output T, conjuct the variables with appropriate negations, then disjunct the lines.
    \item To construct a CNF form, for each line with output F, conjuct the variables with appropriate negations, disjunt them, negate the entire statement. 
    \item Do not repeat symbols for variables and constants.
\end{enumerate}


\end{document}