\documentclass[11pt, cyan, night, 0.5in]{hw}

\def\course{CSC240}
\def\headername{Lecture 11}
\def\name{Joseph Siu}
\def\email{joseph.siu@mail.utoronto.ca}
\def\logo{clsfiles/qunwang}

\usepackage{clsfiles/csc240}

\begin{document}

\coverpage[clsfiles/stars]

\newn{
    Midterm EX310 Wednesday Mar. 6.

    Friday Mar 8 lecture instead of tutorial.

    Offices hours will start $\approx$ 6pm today. 
}

\newtbox[Asymptotic Notation]{
    Let $\cal{F}$ denote the set of all fractions from $\N$ to $\R^+\cup\{0\}$.

    For all $f\in\cal{F}$, let \[O(f)=\{g\in\cal{F}\mid \exists c\in\R^+.
    \exists b\in\N.\forall n\in\N.\sqrbra{(n\geq b)\iimplies (g(n)\bf{\leq} c f(n))}\}\]

    For all $n$ sufficiently large, $g(n)$ is \underline{at most} a constant factor times $f(n)$.

    \[\Om(f)=\{g\in\cal{F}\mid \exists c\in\R^+.\exists b\in\N. \forall n\in\N\sqrbra{(n\geq b)\iimplies (g(n)\bf{\geq} c f(n))}\}\]

    For all $n$ sufficiently large, $g(n)$ is \underline{at least} a constant factor times $f(n)$.

    \neweg{
        $6n+4\in O(3n)$ Since $6n+4\leq 3\times 3n$ for all $n\geq 2$. 
    }

    \newr{
        We use $O(n^2)$ instead of $O(f(n))$ where $f(n)=n^2$ for all $n\in\N$. 
    }

    \[(g\in\Om(f))\iif (f\in O(g))\]

    \[\Th(f)=\{g\in\cal{F}\mid\exists c\in\R^+.\exists c'\in\R^+.\exists b\in\N.\forall n\in\N. (n\geq b)\iimplies (c f(n)\leq g(n)\leq c' f(n))\]
}

\section{Analysis of Algorithms}

For a particular algorithm $A$,

let $t_A(I)$ be the number of steps the algorithm $A$ performs on input $I$. 

\neweg{
    Linear Search $(L,x)$: 
}
\begin{code}
i <- 1
while i <= length(L) do 
    if L[i] = x then return i
    i <- i+1
return 0
\end{code}

Steps:
\begin{itemize}
    \item +1
    \item comparison with $x$ \tbf{(we will focus on this one)}
    \item array access $L(i)$
    \item return \tbf{(this is a bad one)}
    \item omparison of $i$ and length$(L)$
\end{itemize}

on input $L=[2,4,6,8]$, 


\begin{align*}
    & && \T{\underline{\# of steps}}\\
    x&=2 && 1\\
    x&=4 && 2\\
    x&=8 && 4\\
    x&=1 && 4\\
\end{align*}

Want $b$ express running time as a function of input size. 

\subsection{Worst case time complexity}

Let $S_n$ denote the set of all input of size $n$ to the algorithm $A$. 

Then $T_A:\N\to\N$ is the worst case time complexity where $T_A(n)=\max\{t_A(I)\mid I\in S_n\}$.

$T_{\T{linear search}}(n)=n$

\subsection{Average case time complexity}

$T'_A:\N\to\R^+\cup\{0\}$ where $T'_A(n)=\T{E}[t_a]$ where the expectation is taken over propability space on $S_n$. 

\[T'_A(n)=\frac{\ds\sum_{I\in S_n} t_A(I)}{|S_n|}\] \tbf{if all elements in $S_n$ are equally likely. }


\section{Wost Case Time Complexity}

Let $u\in\N\to\N$

Algorithm $A$ has worst case time complexity at most $u$ means $T_A\leq u$.

That is, \[\forall n\in\N. \max\{t_A(I)\mid I\in S_n\}\leq u(n)\]

Or, \[\forall n\in\N.\bf{\forall} I\in S_n. \sqrbra{t_A(I)\leq u(n)}\]

To prove $T_A\leq u$, you must show for all $n\in\N$ and \tbf{for all inputs $I\in S_n$}, algorithm $A$ on input $I$ takes at most $u(n)$ steps. 

\fig[width=0.75\textwidth]{img/2024-02-28-15-55-45.png}

$\ell:\N\to\N$

Algorithm $A$ has worst case time complexity at least $\ell$ means $I_A\geq \ell$

\[\forall n\in\N.\sqrbra{\max\{t_A(I)\mid I\in S_n\}\geq \ell(n)}\]

Or, \[\forall n\in\N. \bf{\exists} I\in S_n. \sqrbra{t_A(I)\geq \ell(n)}\]

\newr{
    This is wrong: In other words, there exists an input $I$ such that $t_A(I)\geq \ell(\T{size}(I))$. 

    We have to show for all $n$. 
}

To prove $T_A\geq l$ for all $n\in\N$, you must \tbf{find an input} $I\in S_n$ such that $A$ takes at least $\ell(n)$ steps on input $I$. 

\np

$T_A\in O(u)$:

\[\exists c\in\R^+. \exists b\in\N.\forall n\in\N. \sqrbra{(n\geq b)\iimplies (T_A(n)\leq c\cd u(n))}\]

Or, \[\exists c\in\R^+.\exists b\in\N.\forall n\in\N.\sqrbra{(n\geq b)\iimplies \forall I\in S_n. t_A(I)\leq c\cd u(n)}\]

$T_A\in\Om(\ell)$:

\[\exists c\in\R^+.\exists b\in\N.\forall n\in\N. (n\geq b)\iimplies (T_A(n)\geq c\cd \ell(n))\]

Or, \[\exists c\in\R^+. \exists b\in\N. \forall n\in\N. (n\geq b)\iimplies \sqrbra{\exists I\in S_n. (t_A(I)\geq c\cd ell(n))}\]

\np

\section{Recursion}

\begin{code}
function square(n)
if n=1
then return n
else return 2*n - 1 + square(n-1)
\end{code}

input size: $n$

step: \# of recursive calls $\times4$= \# of arithmetic operations.

Let $T_{\T{SQ}}(n):\Z^+\to\N$ denote the \# of arithmetic operations performed by square($n$).

\[T_{\T{SQ}}(n)=\begin{cases}
    0 & \T{if } n=1\\
    4+T_{\T{SQ}}(n-1) & \T{if } n>1
\end{cases}.\]

\newd{1}{
    A \underline{recurrence} is a recursively defined function. \underline{Solving a recurrence} means finding a non-recursive description for the function.  
}

\underline{Methods}
\begin{enumerate}[start=0]
    \item Look it up online or in a book.
    \item Guess and Verify.
    \begin{itemize}
        \item Generate a table of values
        \item Look for a pattern
        \item Guess a solution
        \item Check solution for small values of $n$
        \item Prove it is correct using induction 
        \fig{img/2024-02-28-16-33-29.png}
    \end{itemize}
    \item Homogeneous Linear Recurrences using Characteristic Polynomials
    \indenv{
        \begin{align*}
            F(0)&=0\\
            F(1)&=1\\
            F(n)&=F(n-1)+F(n-2), \T{ for } n\geq 2
        \end{align*}

        Constructor Case is a linear combination of a fixed number of preceding terms. 

        \[f(n)=\sum_{i=1}^d a_i f(n-i)\]

        guess $f(n)=c\cd x^n$ where $c,x$ are parameters whose values are chosen letter.

        In this case, \begin{align*}
            cx^n&=cx^{n-1}+cx^{n-2}\\
            x^2&= x+1 \quad\T{This is called the characteristic equation}\\
            x^2-x-1&=0\\
            x&=\frac{1\pm\sqrt{5}}{2}\\
            F(n)=c\cd\bra{\frac{1+\sqrt5}{2}}^n &\T{ or } F(n)=c\cd\bra{\frac{1-\sqrt5}{2}}^n
        \end{align*}

        If $f(n)=y^n$ and $f(n)=x^n$ are two solutions for the self-referential part, then $f(n)=cy^n+c'x^n$ is also a solution.

        Since $0=F(0)=c+c'$ and $1=F(1)=c\bra{\frac{1+\sqrt5}{2}}+c'\bra{\frac{1-\sqrt5}{2}}$. We can solve for $c$ and $c'$, it turns out that $c=\frac1{\sqrt5}$ and $c'=-\frac1{\sqrt5}$.

        Hence, \[F(n)=\frac{\bra{\frac{1+\sqrt5}{2}}^n}{\sqrt5}-\frac{\bra{\frac{1-\sqrt5}{2}}^n}{\sqrt5}\]

        This can br proved by induction. 

        \newr{
            $x^d-\ds\sum_{i=1}^d a_i x^{d-i}=0$ is the characteristic equation.

            If this equation has $d$ distinct roots $r_1,\ldots,r_d$

            $f(n)=\sum_{i=1}^d c_i r_i^n$
        }
    }
    \item Repeated Substitution and Verify (Plug and Chug)
    \indenv{
    
        Example.
        
        For $n\in\Z^+$, let \[M(n)=\begin{cases}
            c & \T{if } n=1\\
            M\bra{\ceil{\frac{n}{2}}}+M\bra{\floor{\frac{n}{2}}}+dn & \T{if } n>1
        \end{cases}.\]

        Consider the special case when $n$ is a power of 2: \[M(n)=\begin{cases}
            c & \T{if } n=1\\
            2M\bra{\frac{n}{2}}+dn & \T{if } n>1
        \end{cases}.\]

        \begin{align*}
            M(n)&=2M\bra{\frac{n}{2}}+dn\\
            &=2\sqrbra{2M(\frac{n}{4})+d\frac{n}{2}}+dn\\
            &=4M(\frac{n}{4})+2dn\\
            &=4\sqrbra{2M(\frac{n}{4})+d\frac{n}{4}}+2dn\\
            &=8M(\frac{n}{8})+3dn\\
            &\vdots\\
            &=2^i M(\frac{n}{2^i})+i dn
        \end{align*}

        When $i=\log_2 n$, we get \begin{align*}
            M(n)&= n\cd M(1) + dn\cd\log_2 n\\
            &= cn + dn \cd \log_2 n
        \end{align*}

        Let $Q(i)=$``$M(2^i)=c2^i+di 2^i$''
        
        We prove $\forall i\in\N. Q(i)$ by induction using the thoerem $M(n)\in\Th(n\log n)$. 
    }
\end{enumerate}

\newt[Master Theorem]{1}{
    Suppose that for $n\in\Z^+$, \[T(c)=\begin{cases}
        c & \T{if } n< B\\
        a_1 T(\ceil{\frac{n}{b}})+ a_2 T(\floor(\frac{n}{b})) + dn^i & \T{if } n\geq B
    \end{cases},\] where $a_1,a_2,B,b\in\N$, $a=a_1+a_2\geq1$, $b>1$, $c,d,i\in\R^+\cup\{0\}$. 

    Then, \[T(n)\in\begin{cases}
        \Th(n^i\log n) & \T{if } a=b^i\\
        \Th(n^i) & \T{if } a<b^i\\
        \Th(n^{\log_b a}) & \T{if } a>b^i
    \end{cases}\]

}


\end{document}